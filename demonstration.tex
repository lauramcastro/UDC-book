%                                                       %
% -- Demo file -- Use as template for your documents -- %
%                                                       %
\documentclass{udc-book}

\usepackage[english]{babel} % Specify language
\addbibresource{references.bib} % Specify BibTeX bibliography file
\graphicspath{{images/}} % Specify directories where pictures are to be found

\begin{document}

%-------------------------------------------------------%
%	TITLE PAGE                                            %
%-------------------------------------------------------%

\printtitlepage{background}{Document Title}{Author Name}
\newpage
\printcopyright{2013}{Author Name}%
{Licensed under the Creative Commons Attribution-NonCommercial 3.0 Unported License (the ``License''). You may not use this file except in compliance with the License. You may obtain a copy of the License at \url{http://creativecommons.org/licenses/by-nc/3.0}.}

%-------------------------------------------------------%
%	TABLE OF CONTENTS
%-------------------------------------------------------%

\chapterimage{chapter_head_1.pdf} % Table of contents heading image
\pagestyle{empty} % No headers
\tableofcontents % Print the table of contents
\cleardoublepage
\pagestyle{fancy} % Print headers again

%-------------------------------------------------------%
%	CHAPTER 1
%-------------------------------------------------------%

\chapterimage{chapter_head_5.pdf} % Chapter heading image

\chapter{Text Chapter}

\section{Paragraphs of Text}\index{Paragraphs of Text}

Blabla bla blablablabla bla, blabla, blabla bla bla blabla blablabla bla blabla
blablablabla.  Blabla bla blablablabla bla, blabla, blabla bla bla blabla
blablabla bla blabla blablablabla.  Blabla bla blablablabla bla, blabla, blabla
bla bla blabla blablabla bla blabla blablablabla.  Blabla bla blablablabla bla,
blabla, blabla bla bla blabla blablabla bla blabla blablablabla.  Blabla bla
blablablabla bla, blabla, blabla bla bla blabla blablabla bla blabla
blablablabla.

Blabla bla blablablabla bla, blabla, blabla bla bla blabla blablabla bla blabla
blablablabla.  Blabla bla blablablabla bla, blabla, blabla bla bla blabla
blablabla bla blabla blablablabla.  Blabla bla blablablabla bla, blabla, blabla
bla bla blabla blablabla bla blabla blablablabla.  Blabla bla blablablabla bla,
blabla, blabla bla bla blabla blablabla bla blabla blablablabla.  Blabla bla
blablablabla bla, blabla, blabla bla bla blabla blablabla bla blabla
blablablabla.

Blabla bla blablablabla bla, blabla, blabla bla bla blabla blablabla bla blabla
blablablabla.  Blabla bla blablablabla bla, blabla, blabla bla bla blabla
blablabla bla blabla blablablabla.  Blabla bla blablablabla bla, blabla, blabla
bla bla blabla blablabla bla blabla blablablabla.  Blabla bla blablablabla bla,
blabla, blabla bla bla blabla blablabla bla blabla blablablabla.  Blabla bla
blablablabla bla, blabla, blabla bla bla blabla blablabla bla blabla
blablablabla.

% - - - - - - - - - - - - - - - - - - - - - - - - - - - %

\section{Citation}\index{Citation}

This statement requires citation \cite{book_key}; this one is more specific
\cite[122]{article_key}.

% - - - - - - - - - - - - - - - - - - - - - - - - - - - %

\section{Lists}\index{Lists}

Lists are useful to present information in a concise and/or ordered
way\footnote{Footnote example...}.

\subsection{Numbered List}\index{Lists!Numbered List}

\begin{enumerate}
\item The first item
\item The second item
\item The third item
\end{enumerate}

\subsection{Bullet Points}\index{Lists!Bullet Points}

\begin{itemize}
\item The first item
\item The second item
\item The third item
\end{itemize}

\subsection{Descriptions and Definitions}\index{Lists!Descriptions and Definitions}

\begin{description}
\item[Name] Description
\item[Word] Definition
\item[Comment] Elaboration
\end{description}

%-------------------------------------------------------%
%	CHAPTER 2
%-------------------------------------------------------%

\chapterimage{chapter_head_3.pdf}

\chapter{In-text Elements}

\section{Definitions}\index{Definitions}

These are examples of definitions.

\subsection{Several definitions}\index{Definitions!Several Points}

\begin{definition}
In $E=\mathbb{R}^n$ all norms are equivalent. It has the properties:
\begin{align}
& \big| ||\mathbf{x}|| - ||\mathbf{y}|| \big|\leq || \mathbf{x}- \mathbf{y}||\\
&  ||\sum_{i=1}^n\mathbf{x}_i||\leq \sum_{i=1}^n||\mathbf{x}_i||\quad\text{where $n$ is a finite integer}
\end{align}
\end{definition}

\subsection{Single Line}\index{Definitions!Single Line}

\begin{definition}[Name]
A set $\mathcal{D}(G)$ in dense in $L^2(G)$, $|\cdot|_0$. 
\end{definition}

% - - - - - - - - - - - - - - - - - - - - - - - - - - - %

\section{Examples}\index{Examples}

This is an example of example.

\begin{example}[Example name]
Given a vector space $E$, a norm on $E$ is an application, denoted $||\cdot||$, $E$ in $\mathbb{R}^+=[0,+\infty[$ such that:
\begin{align}
& ||\mathbf{x}||=0\ \Rightarrow\ \mathbf{x}=\mathbf{0}\\
& ||\lambda \mathbf{x}||=|\lambda|\cdot ||\mathbf{x}||\\
& ||\mathbf{x}+\mathbf{y}||\leq ||\mathbf{x}||+||\mathbf{y}||
\end{align}
\end{example}

% - - - - - - - - - - - - - - - - - - - - - - - - - - - %

\section{Remarks}\index{Remarks}

This is an example of a remark.

\begin{remark}
The concepts presented here are now in conventional employment in mathematics. Vector spaces are taken over the field $\mathbb{K}=\mathbb{R}$, however, established properties are easily extended to $\mathbb{K}=\mathbb{C}$.
\end{remark}

% - - - - - - - - - - - - - - - - - - - - - - - - - - - %

\section{Further Information}\index{Further Information}

This is an example of further information, warnings or such.

\begin{furtherinfo}[Trigger warning]
The concepts presented here are now in conventional employment in mathematics. Vector spaces are taken over the field $\mathbb{K}=\mathbb{R}$, however, established properties are easily extended to $\mathbb{K}=\mathbb{C}$.
\end{furtherinfo}

% - - - - - - - - - - - - - - - - - - - - - - - - - - - %

\section{Exercises}\index{Exercises}

This is an example of an exercise.

\begin{exercise}
This is a good place to ask a question to test learning progress or further cement ideas into students' minds.
\end{exercise}


%-------------------------------------------------------%
%	CHAPTER 3
%-------------------------------------------------------%

\chapterimage{chapter_head_4.pdf}

\chapter{Presenting Information}

\section{Table}\index{Table}

\begin{table}[h]
\centering
\begin{tabular}{l l l}
\toprule
\textbf{Treatments} & \textbf{Response 1} & \textbf{Response 2}\\
\midrule
Treatment 1 & 0.0003262 & 0.562 \\
Treatment 2 & 0.0015681 & 0.910 \\
Treatment 3 & 0.0009271 & 0.296 \\
\bottomrule
\end{tabular}
\caption{Table caption}
\end{table}

% - - - - - - - - - - - - - - - - - - - - - - - - - - - %

\section{Figure}\index{Figure}

\begin{figure}[h]
\centering\includegraphics[scale=0.5]{placeholder}
\caption{Figure caption}
\end{figure}

%-------------------------------------------------------%
%	BIBLIOGRAPHY
%-------------------------------------------------------%

\chapterimage{chapter_head_2.pdf}
\chapter*{Bibliography}
\addcontentsline{toc}{chapter}{\textcolor{mainColor}{Bibliography}}
\section*{Books}
\addcontentsline{toc}{section}{Books}
\printbibliography[heading=bibempty,type=book]
\section*{Articles}
\addcontentsline{toc}{section}{Articles}
\printbibliography[heading=bibempty,type=article]

%-------------------------------------------------------%
%	INDEX
%-------------------------------------------------------%

\cleardoublepage
\setlength{\columnsep}{0.75cm}
\addcontentsline{toc}{chapter}{\textcolor{mainColor}{Index}}
\printindex

\end{document}
